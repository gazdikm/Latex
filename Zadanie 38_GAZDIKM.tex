%79531 Gazdik Martin, FIIT, STU, Predmet MIP, Ing. Jakub �imko, PhD. Ing. Robert Moro
%Odovzdanie 1
\documentclass{llncs}
%! zaciatok hlavicky
\usepackage[cp1250]{inputenc}  %riesi znaky napr. aj slovenskej abecedy, dlzne, makcene
%\usepackage[slovak]{babel} - zakomentovane, pretoze tento dokument je v anglictine
% komentare oznacene  so znakom !tesne za % a teda "%!"sluzia pre neskorsiu korekciu.
%cez Ctrl+F nachadzam teda casti, ktore neskor upravim

\begin{document}

\begin{frontmatter} %hlavicka dokumentu
\title{News Article Classification Based on a Vector Representation Including Words� Collocations}
\author{Michal Kompan and M�ria Bielikov�}
\institute{
Slovak University of Technology, Faculty of Informatics and Information Technologies\\%\\ je newline
Ilkovi�ova 3, 842 16 Bratislava, Slovakia\\
\{kompan, bielik\}@fiit.stuba.sk
}
\maketitle %prikaz na tvorbu hlavicky
 
\begin {abstract}  In this paper we present a proposal including collocations into the
pre-processing of the text mining, which we use for the fast news article
recommendation and experiments based on real data from the biggest Slovak
newspaper. The news article section can be predicted based on several article�s
characteristics as article name, content, keywords etc. We provided experiments
aimed at comparison of several approaches and algorithms including expressive
vector representation, with considering most popular words collocations
obtained from Slovak National Corpus.

\begin{keywords}
text pre-processing, news recommendation, news classification, vector representation
\end{keywords}
\end{abstract} %keywords su v abstrakte kvoli zarovnaniam

\end {frontmatter}
%zaciatok tela
\section{Introduction and related work}
Nowadays no one is discussing the need for the web personalization. One of the ways
in which personalization is performed represents recommendations. Recommendation
task can be defined as follows:\\ %toto bude odstranene ked sa doplni vzorec
%!toto je miesto pre vzorec
where C represent users, S represents objects of recommendation and u is the
usefulness function (usefulness of an object for specific user).

Recommender systems had become important part of well-known web portals in
several domains as online shops, libraries or news portals for years. News portals are
characteristic with thousands of daily added articles with high information decrease
degree, so the one from most relevant recommender systems� attributes are reaction
time of processing and the start of recommending new articles.

There are two widely used approaches to the recommendation: collaborative
filtering based on an assumption that interest is shared between similar users and the
content based recommendation where computed �associations� between entities
(generally similarity relation) based on extracted useful information on the entity
content are used. We focus on the content based recommendation in news domain.

Several recommender systems in the news domain (OTS, NewsMe, Pure, Google
News, NewsBrief [1]) have been proposed in the last decade. The collaborative
recommender on SME.SK [10] and the content based recommenders TRecom [12]
and Titler [7] have been proposed for the Slovak language.
\subsection{Classification task}
Content based recommendation obviously includes recommended entities
classification aimed at finding relations between entities. Considering recommending
news articles, we are often limited to a specific source of articles � news portal. In
such a portal often every article has its own category (mostly assigned by a human),
which is rich and reliable source of important data (in the context of recommendation
and similarity search). Nowadays researchers focus on aggregating recommender
systems like NewsBrief or Google News, where several news portals over the world
are monitored and used for generating recommendations. One of the possible
solutions for the aggregation news from several portals is the classification based on
articles categories from various portals. Several classifiers for news articles have been
proposed respectively [8].

Main goal of the document classification is to assign one or more (probability)
categories of the classified document. In the literature the classification task is often
divided into supervised and unsupervised methods [4], where an unsupervised method
refers to the document clustering. Unsupervised methods are based on an assumption
that documents having similar content should be grouped into the one cluster.
Hierarchical clustering algorithms have been intensively studied as well as neural
network models. Several methods are used as Support vector machines, Na?ve Bayes
or Neural networks [11], where Na?ve Bayes outperforms the others [2].
\subsection{Text representation for classification}
Because of the high information value decrease and the high dynamics in the news
domain, there is need to process objects of the recommendation (articles) in a fast
manner. For this purpose several text representations have been proposed [5].

The simplest method for representation of an article is the Bag of Words (BoW).
As the best unit for the text representation is a term [5], BoW consists of term from
the text. Other often used method of text representation is Vector Space Model, which
adds weights (term frequency) to terms from BoW. In this way we obtain vectors
representing text documents. It is clear that these representations have a huge problem
with dimensionality and thus with the performance of any information retrieval
method applied on. Various enhancements have been proposed as binary
representation, ontology models or N-Grams [5].

Some methods do not consider all terms extracted from texts, but only the relevant.
For the keywords or relevant term extraction Latent Semantic Indexing is often used.
When extracting terms from semi-structured documents (such as HTML) additional
information is used as HTML tags [5] for the relevant terms recognition.
\section{News article representation proposal}
In the domain of news recommendation, the time complexity of the classification
process is critical. To reduce the space of words and to extract relevant information
from articles often a vector representation of text is used. This brings usual the words� space reduction and accuracy improvements (if information extraction is included).
We have proposed a vector representation (Table 1) based on the extraction of
important (distinctive) terms from the article, in order to reduce the dimension of the
space of words. \\
%!miesto pre table 1
The article vector consists of six parts:
\begin{description}
\item [-] \textit{Title} - Article vector comprises lemmatized words from article title. It consists of
approximately 5 words (150 000 Slovak article dataset). We suggest that article
title should be in most occurrences good describing attribute.
\item [-] \textit{Term Frequency of title words in the content} - We used TF to estimate the article
name confidence. If the article name is abstract and do not correspond to article
content, we can easily discover this situation.
\item [-] \textit{Keywords} � We store 10 most relevant keywords. News portals have a list of
keywords for every article usually. These are unfortunately at different abstraction
level over various portals thus we have our own keywords list, which is based on
TF-IDF list calculated over the dataset (100 000 Slovak news articles SME.SK).
\item [-] \textit{Names/Places} - In this step we extract list of names and places obtained from the
article content - as words starting with upper letter and with no full-stop before
(precision = 0.934, recall = 0.863).
\item [-] \textit{CLI} - Coleman-Liau readability index provides information about the level of the
understandability of the text.
%description a item podla youtube tutorialu, textit kvoli predlohe
\end{description}
Most of the vector parts do not depend on a particular news portal, and can be easily
extracted from a standard article. The only one dependent part in our representation is the Category. If we want to abstract of this, it is necessary to find similar articles over
various portals and then respectively create corresponding virtual categories (the text
classification task).
\subsection{Text pre-processing and collocations}
The text pre-processing plays critical role in the text classification process. It can
significantly reduce space of words, but on the other hand, it can easily decrease the
information value. In our experiments we work with texts in Slovak language.
The article pre-processing can be divided into several steps [9]:
\begin{description}
\item [-] \textit{Tokenization} � A simple strategy is to just split the text on all non-alphanumeric
characters. As far as this step is language depending, some information (special
addresses, names etc.) can be lost. Thus, advanced techniques are need for text
tokenization, considering local habits.
\item [-] \textit{Dropping common terms: stop-words} � For every language we can easily identify
most common words without any or only with small information value (and, is, be,
in etc.) By removing these words we are able to significantly reduce the words�
space while in the most cases the information value of processed texts remains.
\item [-] \textit{Normalization} � In other words the process of creating equivalence classes of
terms. The goal is to map words with the same sense to the one class (e.g. �USA�
and �U.S.A.�).
\item [-] \textit{Stemming and lemmatization} � Documents contain different words� forms and
there are families of related words with similar meanings (car, cars, car's, cars' -
car). For the English language the most common stemmer is Porter Stemmer, for
flexive languages such as Slovak language it is more complicated.
\end{description}
As a result of the pre-processing step (in connection to the needs of our vector
representation) we obtain:
- lemmatized article title (without stop words and punctuation),
- 10 most relevant keywords, the list of Names and Places.\\
%! nasleduje nezvycajne odsadenie textu kvoli predlohe (\newline)
\\
The most frequent words occurred in specific language together are considered as
collocations (bigrams). In order to improve the pre-processing step and to increase the
information gain, we introduce words collocations into the text pre-processing step.
We expect that enhancing the pre-processing step with words collocations will lead to
the article similarity or classification tasks improvement.

We extracted word collocations from the Slovak national corpus (�. �t�r Institute
of Linguistic, Slovak Academy of Sciences). The example of collocations for word
�conference� in Slovak are �central�, �OSN�, �focused�. The most frequent
collocations in general are stop words or punctuations. We do not consider such
words (�in the�, �does not� etc.).

In other words we enhanced pre-processing step while not only stop words, but
collocations are removed. This leads to word space reduction. Our hypothesis is that
after removing the collocations the information value of the pre-processed text
remains the same � the information gain (words with distinctive characteristics) will
remain and classification task accuracy will not decrease.
\section{Hypothesis and design of experiments}
Our hypothesis is that introducing words� collocations (removing collocations in the
pre-processing step) can improve classification task results over the dataset. Thus we
suggest several experiments, with various initial settings and classification algorithms.

For the classification task experiments we use SME.SK dataset from the project
SMEFIIT [3]. We have total of 1 387 articles from 20 categories (extracted directly
from news portal) in our dataset. Each article consists of the title, the article content
and the real section in which was assigned by the article author. For each article we
constructed representative article vector as described in Section 2. For the
implementation and experiments we used RapidMiner [6] as one of the well-known
and widely used information discovery environment.

First, we investigated which one from weighting techniques performs best. For this
purpose we modelled a standard classification task with Na?ve Bayes, K-NN and
Decision trees as a classificatory. Weights for words were step by step calculated as
TF-IDF, Term frequencies, Term occurrences and Binary term occurrences. We also
used a pruning method where all words with the weight below 3.0 or above 30.0
percent were pruned. The pruning has negative impact on the whole process
computation complexity. However, this can be compensated by the proposed vector
representation.

In the second experiment we investigated which of the classifiers perform best for
the classification task. We considered K-nearest neighbour, Na?ve Bayes and
Decision trees and their implementations in RapidMiner. For these methods we also
evaluated the best weight function and these best-performers were used in next
experiments.

Our aim was to evaluate properties of proposed article representation for
classification task. So we performed all the experiments for both classical (TF-IDF,
Term Frequency, Term Occurrences, Binary Term Occurrences) and our proposed
representations. We do not use whole vector representation - the Category part was
excluded (it is used as a learner for supervised learning).

Similarly, all experiments were performed for standard pre-processing as we
described in the section 2 (without collocation remove/add). As the next step we
added words� collocations to both representations and we studied performance
changes. In the next experiment collocation were removed instead of added. Because
most frequent collocations are stop words or words with a small information value,
we decided to pre-process these words� collocations too and not to include most
frequent collocations.

\section{Results of experimental evaluation and discussion}
For each experiment we measured the classification accuracy (ration between
correctly and incorrectly classified articles) for pruned and not pruned data
respectively. The evaluation was performed as �X-validation� (x=10) with stratified
sampling. In Table 2 we present the weight functions� comparison for three
classifiers.

As we can see in almost all cases classification performs the best when pruning
method was active. Only in the classification with Na?ve Bayes and with the standard
representation no pruning outperformed the pruned data. The difference between
pruned and not pruned data is significant. It is important to note, that while not
pruning do not brings better results as pruning in general it also takes almost 10x
longer as a classification with pruning.
%!tabulka 2
\\
%! medzera kam pride tabulka
Our proposed vector representation significantly outperforms standard article
representation in the classification accuracy (Fig. 1). We can say that our proposed
representation extracts relevant information (words with distinctive characteristics)
and can be used for various information retrieval tasks not only for the similarity
computation.
%! graf
\\
%! miesto kam pride graf
\\
%! miesto pre text:Fig. 1. Classifiers comparison (Classification accuracy) with various pre-processing methods (TF-IDF, Term Frequency, Term Occurrences, Binary Term Occurrences).
The highest accuracy increase, that we can observe, is using Decision Trees, which
seems to be the best classifier for our task (Average improvement 75,22\%). On the other hand %v tomto riadku sa nachadza v texte znak "%" preto som dal "\" pred neho aby to nechapal latex ako komentar
classification with Decision trees takes the longest time even in case of
vector representation and pruning included. Results when using Decision tree are
�flat� in comparison to other approaches. This can be explained by the used approach,
when the computed weights were not used.

We provided experiments for every possible combination of a weight function, the
classifier, the representation and also considering words� collocations. Because of
similar patterns we do not provide complete results for the collocations excluded.
Aggregated results (mean of 4 weight functions) for the classification with
collocations consideration can be seen in Fig. 2.
%! graf 2
\\
%! miesto pre fig 2 s textom: Fig. 2. Classifiers comparison with collocation consideration (average classification accuracy for 4 weight functions).
Our hypothesis appears to be wrong, i.e. excluding collocations in the pre-processing
step does not significantly improve classification task. However, we can see, that
removing collocations did not degrade the classification accuracy while it reduces
words� space, in other words, correlated words were removed. For the similarity
computation task it can be interesting to experiment with word collocations adding. In
this case the sub-group without stop words and low information gain words should be
carefully selected.
\section{Conclusions}
In this work we compared several classification methods applied to news articles
considering proposed vector representation. The article classification allows us to
abstract from concrete news portal and to start recommending and aggregating articles
from various portals.

We enhanced pre-processing process by introducing words� collocations
excluding. The proposed vector representation outperforms standard representation not only in the way of the classification accuracy (the best improvement 77,27\%) but
it reduces the computation complexity of the classification process which is strictly
connected to the computation time. Such a representation and category classification
also are language independent. When we will replace collocations statistics and stop
words list, we are able to use our proposed method for other languages.

Introducing collocations to the process of pre-processing does not bring
improvement of pre-processing. On the other hand, collocations reduce the word
space while do not decrease the information value at the same time.

Proposed vector representation can be used for content-based news
recommendation and also to aggregate news articles from various news portals (using
category classification) in a fast and effective way. As there are only few language
depending steps during the pre-processing process, various languages can be included
respectively.

%!miesto pre references



\end{document}